\documentclass[]{article}
\usepackage[T1]{fontenc}
\usepackage{lmodern}
\usepackage{amssymb,amsmath}
\usepackage{ifxetex,ifluatex}
\usepackage{fixltx2e} % provides \textsubscript
% use upquote if available, for straight quotes in verbatim environments
\IfFileExists{upquote.sty}{\usepackage{upquote}}{}
\ifnum 0\ifxetex 1\fi\ifluatex 1\fi=0 % if pdftex
  \usepackage[utf8]{inputenc}
\else % if luatex or xelatex
  \usepackage{fontspec}
  \ifxetex
    \usepackage{xltxtra,xunicode}
  \fi
  \defaultfontfeatures{Mapping=tex-text,Scale=MatchLowercase}
  \newcommand{\euro}{€}
\fi
% use microtype if available
\IfFileExists{microtype.sty}{\usepackage{microtype}}{}
\ifxetex
  \usepackage[setpagesize=false, % page size defined by xetex
              unicode=false, % unicode breaks when used with xetex
              xetex]{hyperref}
\else
  \usepackage[unicode=true]{hyperref}
\fi
\hypersetup{breaklinks=true,
            bookmarks=true,
            pdfauthor={},
            pdftitle={},
            colorlinks=true,
            urlcolor=blue,
            linkcolor=magenta,
            pdfborder={0 0 0}}
\urlstyle{same}  % don't use monospace font for urls
\setlength{\parindent}{0pt}
\setlength{\parskip}{6pt plus 2pt minus 1pt}
\setlength{\emergencystretch}{3em}  % prevent overfull lines
\setcounter{secnumdepth}{0}

\author{}
\date{}

\begin{document}

{
\hypersetup{linkcolor=black}
\setcounter{tocdepth}{3}
\tableofcontents
}
\section{Minim checklist description}

This document describes elements of the Minim checklist model. Examples
are presented using Turtle syntax (@@ref).

\subsection{Checklist evaluation context}

The Minim checklist model describes a set of requirements to be
satisfied by some set of linked data for a designated resource to be
suitable for some purpose.

Thus, the context required for evaluating a Minim checklist is a target
resource, a set of linked data (an RDF graph) containing metadata about
the target resource and related resources, and a purpose for which the
resource is being evaluated. The environment or service that invokes a
checklist evaluation must supply:

\begin{itemize}
\itemsep1pt\parskip0pt\parsep0pt
\item
  \emph{metadata}: a set of linked data in the form of an RDF graph,
\item
  \emph{target}: the URI of a resource to be targeted by the evaluation,
  and
\item
  \emph{purpose}: a string that designates the purpose of the evaluation
\end{itemize}

For example, when the evaluation is applied to a Research Object
(@@ref), the target resource may be the Research Object itself, the set
of linked data is the set of annotations aggregated by the Research
Object. But the evaluation may also target an individual resource other
than the Research Object while still depending on context that it
provides, in which case the Research Object and the target resource must
be specified as separate resources.

This information is used in conjunction with a set of
\texttt{minim:Checklist} descriptions (see below) to select a
\texttt{minim:Model} as the basis for the evaluation, and to construct
an initial Minim \emph{evaluation context}. The Minim evaluation context
consists of a set of variable/value bindings; the initial context
contains the following bindings, and maybe others:

~~

targetres

:

the URI of the resource that is the target of the current evaluation

Variables defined in the minim evaluation context, referred to later as
\emph{Minim environment variables}, are used in a number of ways:

\begin{itemize}
\itemsep1pt\parskip0pt\parsep0pt
\item
  In query patterns, query variables that are the same as Minim
  environment variables are treated as being pre-bound: only those query
  results for which the returned variable binding would be the same as
  the existing Minim environment variable binding are returned.
\item
  The results of a query are used as additional variables in constructs
  that depend on the result of that query.
\item
  URI templates are expanded using Minim environment variables to supply
  values for template variables.
\item
  Diagnostic messages containing Python-style \texttt{\%(name)s}
  constructs have those constructs replaced by the value of the named
  Minim environment variable.
\end{itemize}

The exact ways in which the Minim evaluation context is used depends
upon the particular requirement being evaluated. Most commonly, it is
used in conjunction with a \texttt{minim:QueryTestRule} (described
later). For example, the following checklist requirement tests for an
\texttt{rdfs:label} value for the target resource of an evaluation:

\begin{verbatim}
:target_labeled a minim:QueryTestRule ;
  minim:query 
    [ a minim:SparqlQuery ; 
      minim:sparql_query "?targetres rdfs:label ?targetlabel ." ] ;
  minim:min 1 ;
  minim:showpass "Target resource label is %(targetlabel)s" ;
  minim:showfail "No label for target resource %(targetres)s" .
\end{verbatim}

This queries the metadata for an \texttt{rdfs:label} applied to the
evaluation target resource (whose URI is defined in the initial
evaluation context as \emph{\texttt{targetres}}, as described above). If
present, the requirement is satisfied and a message containing the label
is returned. Otherwise, the requirement is not satisfied and a message
containing the target resource URI is returned.

\subsection{Checklist description structure}

A Minim checklist contains 3 levels of description:

\begin{itemize}
\itemsep1pt\parskip0pt\parsep0pt
\item
  \texttt{minim:Checklist} associates a target (e.g.~an RO or a resource
  within an RO) and purpose (e.g.~runnable workflow) with a minim:Model
  to be evaluated.
\item
  \texttt{minim:Model} enumerates the requirements (checklist items) to
  be evaluated, with provision for MUST / SHOULD / MAY requirement
  levels cater for limited variation in levels of checklist conformance.
\item
  \texttt{minim:Requirement} is a single requirement (checklist item),
  which is associated with a rule for evaluating whether or not it is
  satisfied or not satisfied. There are several types of rules for
  performing different types of evaluation of the supplied data.
  Additional evaluation capabilities can be added by adding new rule
  types to the set of those supported.
\end{itemize}

These three levels are called out in the sections below. A definition of
the Minim ontology can be found at
\href{}{https://github.com/wf4ever/ro-manager/tree/develop}

A minim checklist implementation evaluates a checklist in some context.
A minimum realization for such a context is the presence of some RDF
metadata describing the resources associated with it. In our
implementation, the context is a Research Object (@@ref), and the
descriptive metadata is a merge of all its aggregated annotations.

\subsection{minim:Checklist}

The goal of our approach is to determine suitability for some user
selected purpose, so there may be several checklists defined, each
associated with evaluating suitability of some data for a different
purpose.

The role of a \texttt{minim:Checklist} resource is associate a target
resource with a \texttt{minim:Model} that describes requirements for the
resource to be suitable for an indicated purpose. A \texttt{minim:Model}
may be associated directly with a target resource that is the subject of
a \texttt{minim:hasChecklist} statement, or may be associated indirectly
via a URI template that is the object of an associated
\texttt{minim:forTargetTemplate} statement. The latter form is is more
flexible as it allows a given \texttt{minim:Modfel} to be more easily
used with multiple resources. This example indicates a
\texttt{minim:Model} that might be used for evaluating whether or not a
workflow contains a runnable workflow:

\begin{verbatim}
:runnable_workflow a minim:Checklist ;
  minim:forTargetTemplate "{+targetro}" ;
  minim:forPurpose "runnable" ;
  minim:toModel :runnable_workflow_model ;
  rdfs:comment """Select model to evaluate if RO contains a runnable workflow""" .
\end{verbatim}

The \texttt{minim:Checklist} structure provides a link between a
\texttt{minim:Model} structure and the context in which it is evaluated.
In our implementation, the evaluation context is provided by a Research
Object and a supplied Minim description resource, which may contain
several checklist definitions as above. Also provided by the Minim
checklist evaluation context are some Minim environment variables:

~~

targetres

:

the URI of the resource that is the target resource to be evaluated. By
default, this is the URI of provided Research Object.

~~

targetro

:

the URI of the research object that provides contextual information
(metadata) about the target resource.

See above for more information about the Minim checklist evaluation
context.

\subsection{minim:Model}

A minim:Model enumerates of a number of requirements which may be
declared at levels of MUST, SHOULD or MAY be satisfied for the model as
a whole to be considered satisfied. This follows an outline structure
for minimum information models proposed by Matthew Gamble (@@ref). Here
is an example of a model that has been used for testing whether a
runnable workflow is present:

\begin{verbatim}
:runnable_workflow_model a minim:Model ;
  rdfs:label "Minimum information for RO containing a runnable workflow"
  rdfs:comment
    """
    This model defines information that must be available in a Research Object containing a runnable workfow
    which in turn may need a Python software interpeter to be available.
    """ ;
  minim:hasMustRequirement :has_workflow_instance ;
  minim:hasMustRequirement :live_workflow ;
  minim:hasMustRequirement :has_workflow_inputfiles" ;
  minim:hasMustRequirement :environment_python .
\end{verbatim}

\subsection{minim:Requirement}

Minim requirements are evaluated using rules. The current implementation
defined two types of rule: a \texttt{minim:QueryTestRule} and a
\texttt{minim:SoftwareEnvRule}, which are described later. If and when
new requirements are encountered that cannot be covered by available
rules, new rule types may be introduced to the model and added to its
implementation.

The basic structure of a requirement is an association between the
identified requirement and its associated evaluation rule; e.g.

\begin{verbatim}
:has_workflow-instance a minim:Requirement ;
  rdfs:label "Workflow instance requirement" ;
  rdfs:comment
    """
    For a Research Object to contain a runnable workflow, a workflow instance must be specified.
    """ ;
  minim:isDerivedBy :has_workflow_instance_rule .
\end{verbatim}

Each requirement takes the form of a function that returns \texttt{true}
(indicating that the requirement is satisfied), or \texttt{false}
(indicating that the requirement is not satisfied). It may also return a
diagnostic string that may be used when reporting the outcome of the
checklist evaluation. This \texttt{true} or \texttt{false} values are
sometimes referred to as \emph{pass} or \emph{fail}.

\subsubsection{minim:QueryTestRule}

This is a ``swiss army knife'' of a rule which in its various forms is
capable of handling most of the checklist requirements we encounter. It
consists of three parts:

\begin{itemize}
\itemsep1pt\parskip0pt\parsep0pt
\item
  a query pattern, which is evaluated against the RDF metadata that
  described the evaluation context (i.e., in our implementation, the
  merged annotations from a Research Object). The result of evaluating
  this query is a list of variable bindings, each of which defines
  values for one or more variable names that appear in the query
  pattern. Any supplied Minim environment variables are treated as
  pre-bound variables, which allows a query to generate results that are
  dependent on the evaluation context.
\item
  (optionally) an external resource to be queried. This value is used to
  probe data that is external to the evaluation context, rather that
  supplied metadata about the target resource. If not specified, the
  supplied evaluation context metadata is used. The external resource is
  specified as a URI template that is expanded using currently defined
  Minim environment variables, and dereferenced to retrieve an RDF graph
  value.
\item
  a test, which analyzes values from the query result and returns a True
  (pass) or False (fail) result. The test may simply examine the
  supplied query result, or may use that result to perform further
  interrogation of resources outside the immediate context; e.g.~testing
  if a web resource mentioned in the supplied metadata is actually
  accessible.
\end{itemize}

The interaction of a \texttt{minim:QueryTestRule} with the evaluation
context environment variables is particularly important to the way that
it can be used. If the query pattern mentions any variables that are
already defined in the evaluation context, those variables are
considered to be pre-bound in the query. That is, only those query
results in which the query variable matches a value equal to the
environment variable value are returned. Further, other query variables
whose values are returned are used as additional environment variables
in the tests that use the query result, and may be used in URI
templates, diagnostic strings or as pre-bound variables in any further
queries that are used.

Examples of query results used in URI tenmplates and diagnostic strings
can be seen in the \texttt{minim:AccessibilityTest} section below. An
example of query results used in a further query can be seen in the
section on \texttt{minim:RuleTest}.

Several different types of query result test are provided, and
additional test types may be added to the model (and implementation) if
existing tests do not provide the required assurances. The various test
types currently defined are described in the following sections.

\paragraph{\texttt{minim:CardinalityTest} (existence test)}

This test looks for a minimum and/or maximum number of distinct matches
of the query pattern. To test for the existence of some information
matching a query, at least 1 result is expected, as in the following
example that tests for the presence of a workflow resource in the
queried metadata (assuming use of \texttt{wfdesc} terms (@@ref) in the
metadata):

\begin{verbatim}
:has_workflow_instance_rule a minim:QueryTestRule ;
  minim:query 
    [ a minim:SparqlQuery ; 
      minim:sparql_query "?wf rdf:type wfdesc:Workflow ." ] ;
  minim:min 1 ;
  minim:showpass "Workflow instance or template found" ;
  minim:showfail "No workflow instance or template found" .
\end{verbatim}

@@need to decide if show/showpass/showfail are part of the requirement
or part of the rule. I think they should be part of the requirement, but
this means we need to define that Minim environment variables can be
returned by rules.

\paragraph{\texttt{minim:AccessibilityTest} (liveness test)}

For each result returned by the query, tests that a resource is
accessible (live). If there is any result for which the accessibility
test fails, then the rule as a whole fails.

Each set of variable bindings returned by the query is used to construct
the URI of a resource to be tested through expansion of a URI template,
where the query variables are mapped to variables of the same name used
in the template.

The following example tests that each workflow definition mentioned in
the queried metadata is accessible. If it is a local file, a file
existence check is performed. If it is a web resource, then a success
response to an HTTP HEAD request is expected.

\begin{verbatim}
:live_workflow_rule a minim:QueryTestRule ;
  minim:query 
    [ a minim:SparqlQuery ; 
      minim:sparql_query 
        """
        ?wf rdf:type wfdesc:Workflow ;
          rdfs:label ?wflab ;
          wfdesc:hasWorkflowDefinition ?wfdef .
        """ ] ;
  minim:isLiveTemplate {+wfdef} ;
  minim:showpass "All workflow instance definitions are live (accessible)" ;
  minim:showfail "Workflow instance defininition %(wfdef)s for workflow %(wflab)s is not accessible" .
\end{verbatim}

\paragraph{\texttt{minim:AggregationTest}}

This test is specific to a Research Object context. It tests to see if a
resource defined by each query result is aggregated by the Research
Object.

The following example tests that each workflow definition mentioned in
the queried metadata is aggregated in the Research Object. The RO URI is
accessible through Minim environment variable \texttt{targetro}, as
described above.

\begin{verbatim}
:aggregated_workflow_rule a minim:QueryTestRule ;
  minim:query 
    [ a minim:SparqlQuery ; 
      minim:sparql_query 
        """
        ?wf rdf:type wfdesc:Workflow ;
          rdfs:label ?wflab ;
          wfdesc:hasWorkflowDefinition ?wfdef .
        """ ] ;
  minim:aggregatesTemplate {+wfdef} ;
  minim:showpass "All workflow instance definitions are aggregated by RO %(targetro)s" ;
  minim:showfail "Workflow instance defininition %(wfdef)s for workflow %(wflab)s is not aggregated by RO %(targetro)s" .
\end{verbatim}

\paragraph{\texttt{minim:RuleTest}}

The variable bindings from each query result are used as additional
Minim environment variables in a new rule invocation. If the new
invocation succeeds for every such result, then the current rule
succeeds.

The following example uses a cardinality test for each workflow
described in the metadata to ensure that each such workflow has at least
one defined input resource:

\begin{verbatim}
:has_workflow_inputfiles a minim:QueryTestRule ;
  minim:query 
    [ a minim:SparqlQuery ; 
      minim:sparql_query 
        """
        ?wf rdf:type wfdesc:Workflow ;
          rdfs:label ?wflab .
        """ ] ;
  minim:affirmRule
    [ a minim:QueryTestRule ;
      minim:query
        [ a minim:SparqlQuery ; 
          minim:sparql_query 
            """
            ?wf wfdesc:hasInput [ wfdesc:hasArtifact ?if ] .
            """ ] ;
      mnim:min 1 ;
      minim:showpass "Workflow %(wflab)s has defined input(s)" ;
      minim:showfail "Workflow %(wflab)s has no defined input" ] ;
  minim:showpass "All workflow instance definitions have defined inputs" ;
  minim:showfail "Workflow %(wflab)s has no defined input" .
\end{verbatim}

\paragraph{\texttt{minim:RuleNegationTest}}

@@is this needed?

As previous, but the current rule succeeds if the referenced rule fails
(forall/forsome?) query result.

\paragraph{\texttt{minim:ExistsTest}}

@@this test is strictly redundant

If defined, this would be a short cut form for \texttt{minim:RuleTest}
with \texttt{minim:CardinalityTest}; e.g.~the previous
\texttt{minim:RuleTest} example might be presented as:

\begin{verbatim}
:has_workflow_inputfiles a minim:QueryTestRule ;
  minim:query 
    [ a minim:SparqlQuery ; 
      minim:sparql_query 
        """
        ?wf rdf:type wfdesc:Workflow ;
          rdfs:label ?wflab .
        """ ] ;
  minim:exists
    [ a minim:SparqlQuery ; 
      minim:sparql_query 
        """
        ?wf wfdesc:hasInput [ wfdesc:hasArtifact ?if ] .
        """ ] ;
  minim:showpass "All workflow instance definitions have defined inputs" ;
  minim:showfail "Workflow %(wflab)s has no defined input" .
\end{verbatim}

\subsubsection{Software environment testing}

A \texttt{minim:SoftwareEnvRule} tests to see if a particular piece of
software is available in the current execution environment by issuing a
command and checking the response against a supplied regular expression.
(This test is primarily intended for local use within RO-manager, and
may be of limited use on the evaluation service as the command is issued
on the host running the evaluation service, not on the host requesting
the service.)

The result of running the command (i.e.~data written to its standard
output stream) is used to define a new Minim environment variable, which
can be used for diagnostic purposes.

\begin{verbatim}
:environment_python a minim:SoftwareEnvRule ;
  minim:command "python --version" ;
  minim:response "Python 2.7" ;
  minim:show "Installed python version %(response)s" .
\end{verbatim}

\section{Notes}

\begin{itemize}
\itemsep1pt\parskip0pt\parsep0pt
\item
  Currently, the OWL ontology does not define the diagnostic message
  prioperties
\item
  Need to decide how diagnostics should be incorporatedL: as part of
  requirement or part of rule?
\item
  Negated rule test; need to think if all or some results should result
  in failure
\item
  Maybe need to think about generalizing \texttt{minim:Ruletest} to
  handle rule composition
\item
  Drop \texttt{minim:existstest}, or keep it?
\item
  See:
  http://www.essepuntato.it/lode/owlapi/https://raw.github.com/wf4ever/ro-manager/develop/Minim/minim-revised.omn
\end{itemize}

\end{document}
